\addsection{Optional Modifications}{\images/wisdom.png}

\textit{This is a set of smaller modifications I designed to improve the standard game, but some of them may work good also with "Dice Stacks" mod. Every section may be applied independently.}

\begin{multicols*}{2}
    
    \subsection*{More \svg{ongoing} Spells}
    
    Every \svg{instant} spell targeting some Attacking or Defending Unit becomes \svg{ongoing} effect until the end of combat round instead. During this round every Attack delivered or received by that Unit gets the Spell's effect.
    
    \begin{itemize}
        \item \emph{Curse} effect is modified: it always reduces \svg{defense} by 1. Spell Power decides what unit may be affected: 0 - \bronze, 1 - \bronze \silver, 2 - \bronze \silver \golden
        \item This works best with \textit{Better AI fight} modification. The default AI may too easily make hopeless attacks or expose to strong retaliations.
    \end{itemize}
    
    \subsection*{Slow/Haste effect}
    
    If \svg{unit_ground} or \svg{unit_flying} unit's initiative is higher/lower than it's base value due to any effect, it may move 1 square more/less.
    
    \subsection*{Better Building Material Trade}
    
    To improve somewhat life of Players getting two \svg{building_materials} Mines on their Far Tiles, you may allow trading 2 \svg{building_materials} for 3 \svg{gold} in addition to standard trades.
    
    \subsection*{Better AI fight}
    
    \textit{These instructions should make AI fight more challenging, without forcing you to think how to play again yourself.}
    
    \subsubsection*{Picking target}
    
    Start with all reachable enemy units and as potential targets. On every step remove from that set units which does not meet the condition. Once only one unit remains in set, it is picked for attack. If any step would remove all units, it is skipped.
    
    
    \begin{itemize}
        \item Units with \svg{attack} higher than \svg{defense} of the activated unit.
        \item Units whose \svg{health_points} will be reduced to $0$ by attack of the activated unit.
        \item \textit{Skip if \svg{unit_ranged} is attacking non-adjacent Unit:} Units which will \emph{not} reduce activated unit's \svg{health_points} to $0$ on retaliation.
        \item \textit{Skip if \svg{unit_ranged} is attacking:} Units which may be attacked from a spot adjacent to some enemy's \svg{unit_ranged} unit.
        \item \textit{Skip if \svg{unit_ranged} is attacking non-adjacent Unit:} Units whose retaliation will give least damage to the activated unit.
        \item Unit picked according to the standard rules.
    \end{itemize}
    
    Once a target is picked, \svg{unit_ground} and \svg{unit_flying} prioritize attacking from spots adjacent to some enemy \svg{unit_ranged}.
   
    When considering results of potential attack or retaliation, take all effects into account.  Assume "0" as die result. If two dices are to be thrown, assume "-1" and "1" results.
    
    \subsubsection*{Unit deployment}
    
    \begin{itemize}
        \item First nominate a \textbf{base flank} for unit's deployment: the side of the battle board, on which half there's less enemy units. If tied, compare number of units of different tiers, starting from the strongest. If still tied, pick side randomly.
        \item Deploy all \svg{unit_ranged} in second row in order of desceding initiative, starting from \textbf{base flank}.
        \item Then deploy one from remaining units, selecting one with higher rank, then initiative. Place it at the first square from the base flank in the first row, and -- if possible -- move it in that row to the nearest spot from where the deployed unit may reach an enemy \svg{unit_ranged} or unit which falls to firts two conditions in \textit{Picking target} subsection above.
        \item Deploy rest of the units in the first row starting from the base flank, in standard rules order, then push them back to second row where possible.
    \end{itemize}
\end{multicols*}
