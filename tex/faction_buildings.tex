\addsection{Faction Buildings}{\images/first_aid.png}

For each faction there is also proposition of new costs and effects of faction-specific buildings, making them stronger and worth considering before building all dwellings.

\hommtable{17}{
    \centering
    \medskip
    \textbf{Dungeon}\\
    \bigskip
    \begin{tabularx}{0.96\linewidth}{XX}
        \darkcell{Portal of Summoning} & \darkcell{Mana Vortex} \\
        \lightcell{4 \svg{gold} 3 \svg{building_materials}} & \lightcell{4 \svg{gold}} \\
        \lightcell[4.7]{At the beginning of your turn, you may draw 1 Neutral Unit card from decks corresponding to the Dwellings in your Town and \svg{pay_v2-note} the Recruitment cost to \textit{Recruit} this unit.} & \lightcell[4.7]{At the beginning of your turn, you may discard 1 card from your hand, to Search (3) from your M\&M deck, optionally shuffling your discard pile back into your deck before.}  \end{tabularx}
}

\hommtable{17}{
    \centering
    \centering
    \medskip
    \textbf{Inferno}\\
    \bigskip
    \begin{tabularx}{0.96\linewidth}{XX}
        \darkcell{Brimstone Stormclouds} & \darkcell{Castle Gate} \\
        \lightcell{2 \svg{gold} 1 \svg{valuables}} & \lightcell{7 \svg{gold} 5 \svg{building_materials}} \\
        \lightcell[4.7]{When built and at the beginning of each round, place your faction cube here (to a maximum of 3). During any Combat, you can remove them to gain +1 \svg{empower-note} per 1 cube. Only one cube can be used per 1 \svg{spellpower-note}.} & \lightcell[4.7]{You may defend your Mine without hero like City or Settlement.
            
            \vspace{1em}
            
            When defending Settlement, City or Mine, you don't have to pay 8 \svg{gold} and you may use M\&M cards during battle.}  \end{tabularx}
}

\hommtable{17}{
    \centering
    \medskip
    \textbf{Castle}\\
    \bigskip
    \begin{tabularx}{0.96\linewidth}{XX}
        \darkcell{Blacksmith} & \darkcell{Brotherhood of the Sword} \\
        \lightcell{4 \svg{gold} 1 \svg{building_materials}} & \lightcell{8 \svg{gold} 4 \svg{building_materials}} \\
        \lightcell[4.7]{\textbf{When build:} Search 2 \svg{artifact-note}
            
            \vspace{1em}
            
            \textbf{Once per turn:} \svg{pay_v2-note} 6 to Search (2) \svg{artifact-note} or remove \svg{artifact-note} from your hand to gain 4 \svg{gold}} & \lightcell[4.7]{At the beginning of each round, gain a \svg{morale_positive-note}}  \end{tabularx}
}

\hommtable{17}{
    \centering
    \medskip
    \textbf{Stronghold}\\
    \bigskip
    \begin{tabularx}{0.96\linewidth}{XX}
        \darkcell{Freelancer's Guild} & \darkcell{Hall of Valhalla} \\
        \lightcell{2 \svg{gold} 1 \svg{building_materials}} & \lightcell{7 \svg{gold} 4 \svg{building_materials}} \\
        \lightcell[4.7]{Each time you win against Neutral Units, gain 1 \svg{gold}. When Reinforcing or Recruiting you can use \svg{building_materials} and \svg{valuables} like \svg{gold}.} & \lightcell[4.7]{Once per round, one of your units gains +2 \svg{attack-note}  to a single attack.}  \end{tabularx}
}

\hommtable{17}{
    \centering
    \medskip
    \textbf{Rampart}\\
    \bigskip
    \begin{tabularx}{0.96\linewidth}{XX}
        \darkcell{Mystic Pond} & \darkcell{Saplings} \\
        \lightcell{6 \svg{gold} 2 \svg{valuables}} & \lightcell{3 \svg{gold} 1 \svg{valuables}} \\
        \lightcell[4.7]{At the beginning of each Resource round, roll 2 \svg{trasuredie-note} and resolve one.} & \lightcell[4.7]{\textbf{When build:} Reinforce Dendroids for free.}  \end{tabularx}
}

\hommtable{17}{
    \centering
    \medskip
    \textbf{Tower}\\
    \bigskip
    \begin{tabularx}{0.96\linewidth}{XX}
        \darkcell{Artifact Merchant} & \darkcell{Wall of Knowledge} \\
        \lightcell{8 \svg{gold} 2 \svg{building_materials} 1 \svg{valuables}} & \lightcell{3 \svg{gold} 2 \svg{building_materials}} \\
        \lightcell[4.7]{\textbf{When build:} Look through \svg{artifact-note} deck until you find two Relics. Take one into your hand and set aside the other.
            
            \vspace{1em}
            
            During your turn you may 
            
            1. \svg{pay_v2-note} 7 to take set-asite Relic to your hand or \textbf{Search (2)} \svg{artifact-note}
            
            2. remove \svg{artifact-note} from hand to gain 3 \svg{gold} per each.} & \lightcell[4.7]{At the beginning of your turn, you may take 1 Knowledge or 1 Power Statistic card from your discard pile to your hand.}  \end{tabularx}
}

\hommtable{17}{
    \centering
    \medskip
    \textbf{Fortress}\\
    \bigskip
    \begin{tabularx}{0.96\linewidth}{XX}
        \darkcell{Blood Obelisk} & \darkcell{Cage of Warlords} \\
        \lightcell{6 \svg{gold} 4 \svg{building_materials}} & \lightcell{4 \svg{gold} 2 \svg{building_materials}} \\
        \lightcell[4.7]{When built and on start of every combat, you may Search(3) your discard pile.} & \lightcell[4.7]{When built and at the beginning of each round, place a faction cube here (to a maximum of 2). During any Combat, a player may remove them to gain +1 \svg{attack-note} or +1 \svg{defense-note} per 1 cube.}  \end{tabularx}
}

\hommtable{17}{
    \centering
    \medskip
    \textbf{Necropolis}\\
    \bigskip
    \begin{tabularx}{0.96\linewidth}{XX}
        \darkcell{Cover of Darkness} & \darkcell{Necromancy Amplifier} \\
        \lightcell{5 \svg{gold} 4 \svg{building_materials} 1 \svg{valuables}} & \lightcell{3 \svg{gold} 1 \svg{building_materials} 1 \svg{valuables}} \\
        \lightcell[4.7]{During your turn, choose one:
            
            \vspace{1em}
            
            1. Draw 2 cards from your M\&M deck, then discard 2 cards from hand.
            
            \vspace{1em}
            
            2. At the beginning of Combat with an Enemy Hero, discard 1 random card from the enemy's hand.} & \lightcell[4.7]{\textbf{When built:} Search the Ability deck for a Necromancy card and put it in your hand.
            
            \vspace{1em}
            
            At the beginning of your turn, Take 1 Specialty or Necromancy card from your discard pile to your hand.}  \end{tabularx}
}
